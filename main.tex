% Ignorierte Warnings
\RequirePackage{silence}
\WarningFilter{chktex}{You should perhaps use `\min' instead.} % unfortunately -
\WarningFilter{chktex}{You should perhaps use `\max' instead.} % - doesn't work

\documentclass[11pt,twocolumn,abstract=true]{scrartcl}
% Documentclass-options

%%%%% Workarounds %%%%%
\setlength{\marginparwidth}{2cm} % without: trouble with todonotes

\newcommand*\appendixmore{% add title to Abstract
  \addsec{\appendixname}%
  \renewcommand{\thesubsection}{\Alph{subsection}}%
}
%%%%% Workarounds %%%%%


% Language-Encoding
\usepackage{polyglossia}        % Alternative zu Babel
\setmainlanguage[variant=british]{english}
\setotherlanguage[babelshorthands=true]{german}


% Citation, Verweise
\usepackage[backend=biber]{biblatex}
\addbibresource{BibLaTex/citation_db.bib}
\usepackage{csquotes}           % Bei Polyglossia+Babel empfohlen
\usepackage{nameref}
\usepackage{listings}           % Paket zum Einfügen von Source Code.
                                % refer to 'minted vs. texments vs. verbments'


% Fonts
\usepackage{fontspec}
\usepackage{glossaries} % \gls{<golssary-entry>} muss vor unicode-math geladen werden
\usepackage{unicode-math} % fontspec wird auch von unicode-math geladen
\setkomafont{disposition}{\normalfont\bfseries}
\setmainfont{TeX Gyre Pagella}
\setmathfont[ItalicFont=*, BoldFont=*]{TeX Gyre Pagella Math}
% \setmainfont{Times New Roman} - Unverhältnismäßig hässlich. Besser keine Font
% setzen.


% Sonstiges                     % Einfügen von Grafiken
\usepackage{graphicx}
\graphicspath{ {./bilder/} }
\usepackage{xcolor}
\usepackage[normalem]{ulem}     % Unterstreichen von Text mit \uline
\usepackage{titling}  
\usepackage{blindtext}          % Repräsentativer als Lorem Ipsum
\usepackage{kantlipsum}         % Cooler als blindtext
\usepackage{todonotes} % Einfügen von Todos, Option: [disable]
\usepackage{url,hyperref}

\author{Malte Zietlow, Anna-Lena Büßelmann}

\begin{document}

\title{Survey: Visual Approaches Towards Interpretability Of Neural Networks}
\maketitle


% Frontmatter
\listoftodos{}            % Aktuelle ToDo-Notes
\begin{abstract}
    \blindtext{1}
\end{abstract}

% Mainmatter
\section{Introduction}
\blindtext[3]
\section{Background}

\blindtext[3]
\section{General Concepts}
\blindtext[3]
\subsection{Types of Explanation}
\blindtext[3]
\subsection{Metrices}
\blindtext[3]
\input{Document-Structure/02_Mainmatter/04_Gradient-Based}
\input{Document-Structure/02_Mainmatter/05_Perturbation-Based}
\input{Document-Structure/02_Mainmatter/06_Findings}
\section{Known Issues}
\blindtext[3]

\section{Conclusion}
\blindtext[3]

\section{Future Research}
\blindtext[3]


% Backmatter
\input{Document-Structure/03_Backmatter/01_Appendices}
\input{Document-Structure/03_Backmatter/02_Bibliography-Glossaries-and-Lists}
\input{Document-Structure/03_Backmatter/03_GlossarMetadaten}

\end{document}