% Ignorierte Warnings
\RequirePackage{silence}
\WarningFilter{chktex}{You should perhaps use `\min' instead.} % unfortunately -
\WarningFilter{chktex}{You should perhaps use `\max' instead.} % - doesn't work

\documentclass[11pt,twocolumn,abstract=true]{scrartcl}
% Documentclass-options

%%%%% Workarounds %%%%%
\setlength{\marginparwidth}{2cm} % without: trouble with todonotes

\newcommand*\appendixmore{% add title to Abstract
  \addsec{\appendixname}%
  \renewcommand{\thesubsection}{\Alph{subsection}}%
}
%%%%% Workarounds %%%%%


% Language-Encoding
\usepackage{polyglossia}        % Alternative zu Babel
\setmainlanguage[variant=british]{english}
\setotherlanguage[babelshorthands=true]{german}


% Citation, Verweise
\usepackage[backend=biber]{biblatex}
\addbibresource{BibLaTex/citation_db.bib}
\usepackage{csquotes}           % Bei Polyglossia+Babel empfohlen
\usepackage{nameref}
\usepackage{listings}           % Paket zum Einfügen von Source Code.
                                % refer to 'minted vs. texments vs. verbments'


% Fonts
\usepackage{fontspec}
\usepackage{glossaries} % \gls{<golssary-entry>} muss vor unicode-math geladen werden
\usepackage{unicode-math} % fontspec wird auch von unicode-math geladen
\setkomafont{disposition}{\normalfont\bfseries}
\setmainfont{TeX Gyre Pagella}
\setmathfont[ItalicFont=*, BoldFont=*]{TeX Gyre Pagella Math}
% \setmainfont{Times New Roman} - Unverhältnismäßig hässlich. Besser keine Font
% setzen.


% Sonstiges                     % Einfügen von Grafiken
\usepackage{graphicx}
\graphicspath{ {./bilder/} }
\usepackage{xcolor}
\usepackage[normalem]{ulem}     % Unterstreichen von Text mit \uline
\usepackage{titling}  
\usepackage{blindtext}          % Repräsentativer als Lorem Ipsum
\usepackage{kantlipsum}         % Cooler als blindtext
\usepackage{todonotes} % Einfügen von Todos, Option: [disable]
\usepackage{url,hyperref}

\author{Malte Zietlow, Anna-Lena Büßelmann}

\begin{document}

\title{Survey: Visual Approaches Towards Interpretability Of Neural Networks}
\maketitle


% Frontmatter
\listoftodos{}            % Aktuelle ToDo-Notes
\begin{abstract}
    \blindtext{1}
\end{abstract}

% Mainmatter
\section{Introduction}
\blindtext[3]

\subsection{Structure of this paper}
\blindtext[3]

\subsection{Target audience}
\blindtext[3]

\subsection{Criteria for inclusion}
\blindtext[3]
\section{Background}

\blindtext[3]
\section{General Concepts}
\blindtext[3]
\subsection{Types of Explanation}
\blindtext[3]
\subsection{Metrices}
\blindtext[3]
\input{Document-Structure/02_Mainmatter/04_Gradient-Based}
\input{Document-Structure/02_Mainmatter/05_Perturbation-Based}
\section{Findings}
\blindtext[3]

\section{Known Issues}
\blindtext[3]

\section{Conclusion}
\blindtext[3]

\section{Future Research}
\blindtext[3]


% Backmatter
\appendix

\section{Basic Concepts}\todo{Exchange for a better title}
\blindtext[1]

\section{Algorithms of related work}
\blindtext[2]

\section{Concepts of Bounding Box Encoding}\label{append:Concepts of Bounding Box Encoding}
\blindtext[3]



\section{K-Means Clustering}\label{append:K-Means Clustering}
\blindtext[3]~\cite[386-390]{James.2017}
\section{Bounding Box Offsets}
\blindtext[1]

\section{Preprocessing}
\blindtext[3]

\section{Concepts of Neural Network Architectures}
\blindtext[3]
\subsection{Convolutional Layer}\label{append:Convolutional Layer}
receptive field~\cite[335-345]{IanGoodfellow.2016}
\blindtext[1]
\blindtext[3]

\section{Loss Functions}
\blindtext[2]

\section{VGG16 and ResNet (Base Networks)}
\blindtext[3]


%\chapter{General Todos}

% Consider colophon  
%% You’re recommended to use the eprint-aware biblio styles which
%% can be obtained from e.g. www.arxiv.org. The file mythesis.bib
%% is derived from the source using the SPIRES Bibtex service.
\onecolumn{\printbibliography{}}  %% Literaturverzeichnis
\printglossaries{}      % Glossar
%% I prefer to put these tables here rather than making the
%% front matter seemingly interminable. No-one cares, anyway!
\listoftables           % Tabellenverzeichnis
\listoffigures          % Abbildungsverzeichnis
\setacronymstyle{long-short}

% ----------------------------- Acronyms -----------------------------
\newacronym{ml} {ML} {Machine Learning}
\newacronym{skcm} {SKCM} {Simple K-Counting Machine}
\newacronym{ai} {AI} {Artificial Intelligence}
\newacronym{rnn} {RNN} {Recurrent Neural Network}
\newacronym{cnn} {CNN} {Convolutional Neural Network}
\newacronym{fcnn} {FCNN} {Fully Convolutional Neural Network}
\newacronym{yolo} {YOLO} {You Only Look Once}
\newacronym{ssd} {SSD} {Single Shot MultiBox Detector}
\newacronym[description={A common activation function for neural neworks.
\(f(x)=\max(0, x)\)}] {relu} {ReLU} {Rectified Linear Unit}

\newacronym[description={Umsetzung eines Features oder Produktes mit zwar
minimalem Funktionsumfang, aber dennoch konkretem Mehrwert für den Nutzer},
plural={MVPs}] {mvp}{MVP}{Minimal Viable Product}

% ----------------------------- Glossary Entries -----------------------------

\newglossaryentry{feature map} {
    name={feature map},
    description={Output of a convolutional layer after applying an activation function like \gls{relu}}
}

\newglossaryentry{anchor} {
    name={anchor},
    description={Center of a bounding box},
}

\newglossaryentry{layer} {
    name={layer},
    description={Contextsensitive, either dense- or \gls{convolutional layer} 
    as defined by in}}

\newglossaryentry{convolutional layer} {
    name={convolutional layer},
    description={Composition of the convolutional operation and nonlinearity, 
    partially referring to the `complex layer terminology' 
    in~\cite[341]{IanGoodfellow.2016}}
}

\newglossaryentry{dense layer} {
    name={dense layer},
    description={Also: fully connected layer. Composition of the matrix operation and nonlinearity, partially referring to the `complex layer terminology' 
    in~\cite[341]{IanGoodfellow.2016}} \todo{refine.~explain also conv in appendix.}
}

\newglossaryentry{deconvolutional layer} {
    name={deconvolutional layer},
    description={Deconvolutional layer. Composition of the deconvolutional 
    operation and nonlinearity, partially referring to the `complex layer 
    terminology' 
    in~\cite[341]{IanGoodfellow.2016}} \todo{refine.~explain also conv in 
    appendix.}
}


\end{document}