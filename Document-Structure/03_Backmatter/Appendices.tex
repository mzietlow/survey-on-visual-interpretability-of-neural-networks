\appendix

%\chapter{General Todos}
\todo{Decisions are non-linear combinations of inputs. Knowing the importance
of each individual input does not tell us about the dynamic relationships between inputs. Furhtermore, `pixels rarely the basic units used in human
image understanding; instead, we would rely on strokes and other higher order features Beyond raw features – feature basis'~\cite{AlvarezMelis.2018}}
\todo{Cite fig.4 from~\cite{AlvarezMelis.2018}}
\todo{Models must be made formal, i.e.\ \(f(x)\)\ldots and the explanation \(f_\text{expl}(x)\) or something similar}
\todo{Normalize: full dataset vs.\ complete dataset}
\todo{Label all equations with ascending index?}
\todo{Confirm that subtitles for conferences are actually cited, i.e.\ ICML 2017 should become `Proceedings of the 34th\ldots' and so on. (Hint: they are not -> therefore, keep it as is and fix when finished)}
\todo{t-SNE was proposed by Maarten and Hinton, `Visualizing Data using t-SNE'}
\todo{Ancona, a unified view of gradient-based attribution methods, propose notation for input-sample x where the i'th pixel was replaced}
\todo{Decide how to treat input-samples, i.e.\ \(x\), and also datasets, i.e.\ \(X\), also against vectors and matrices \(\symbfit{x}, \symbfit{X}\)}
\todo{decide between input-sample and input-example}
\todo{rename propagation-methods to projection-methods? at least the action projecting fits better than propagating, i guess.}
\todo{Differentiate between gradient and projection/propagation based methods}