\setacronymstyle{long-short}

% ----------------------------- Acronyms -----------------------------
\newacronym{ml} {ML} {Machine Learning}
\newacronym{skcm} {SKCM} {Simple K-Counting Machine}
\newacronym{ai} {AI} {Artificial Intelligence}
\newacronym{rnn} {RNN} {Recurrent Neural Network}
\newacronym{cnn} {CNN} {Convolutional Neural Network}
\newacronym{fcnn} {FCNN} {Fully Convolutional Neural Network}
\newacronym{yolo} {YOLO} {You Only Look Once}
\newacronym{ssd} {SSD} {Single Shot MultiBox Detector}
\newacronym[description={A common activation function for neural neworks.
\(f(x)=\max(0, x)\)}] {relu} {ReLU} {Rectified Linear Unit}

\newacronym[description={Umsetzung eines Features oder Produktes mit zwar
minimalem Funktionsumfang, aber dennoch konkretem Mehrwert für den Nutzer},
plural={MVPs}] {mvp}{MVP}{Minimal Viable Product}

% ----------------------------- Glossary Entries -----------------------------

\newglossaryentry{feature map} {
    name={feature map},
    description={Output of a convolutional layer after applying an activation function like \gls{relu}}
}

\newglossaryentry{anchor} {
    name={anchor},
    description={Center of a bounding box},
}

\newglossaryentry{layer} {
    name={layer},
    description={Contextsensitive, either dense- or \gls{convolutional layer} 
    as defined by in}}

\newglossaryentry{convolutional layer} {
    name={convolutional layer},
    description={Composition of the convolutional operation and nonlinearity, 
    partially referring to the `complex layer terminology' 
    in~\cite[341]{IanGoodfellow.2016}}
}

\newglossaryentry{dense layer} {
    name={dense layer},
    description={Also: fully connected layer. Composition of the matrix operation and nonlinearity, partially referring to the `complex layer terminology' 
    in~\cite[341]{IanGoodfellow.2016}} \todo{refine.~explain also conv in appendix.}
}

\newglossaryentry{deconvolutional layer} {
    name={deconvolutional layer},
    description={Deconvolutional layer. Composition of the deconvolutional 
    operation and nonlinearity, partially referring to the `complex layer 
    terminology' 
    in~\cite[341]{IanGoodfellow.2016}} \todo{refine.~explain also conv in 
    appendix.}
}
