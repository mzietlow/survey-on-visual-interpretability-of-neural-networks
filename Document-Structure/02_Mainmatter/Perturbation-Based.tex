\section{Perturbation-Based Methods}
Perturbation based methods for Visualizing NNs are methods, which compute “the attribution of an input feature (or set of features) by removing, masking or altering them, and running a forward pass on the new input, measuring the difference with the original output.” (Acona et al., p.2).
These methods “allow a direct estimation of the marginal effect of a feature” (Acona et al., p.2)., but do not achieve the needed performance when it comes to a higher number of features.
Because of this, there are many new approaches in this field.
There are various methods which are based on perturbation of input data. In this meta study the focus will be on Prediction Difference Analysis by Robnik-Šikonja & Kononenko (2008), but multiple other methods will be mentioned and and the current state of research will be explained.