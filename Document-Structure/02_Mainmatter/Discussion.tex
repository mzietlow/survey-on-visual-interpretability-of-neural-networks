\section{Conclusion}
Concluding this literature-review on visual-interpretability of non-linear predictors (with significant focus on \glspl{nn}), this area of research feels more like a minefield, where every few months, new work is published that either outperforms previous methods or renders them flat wrong. With interpretability as a requirement for certain application domains (especially in the european union), cooling is improbable. Interpretability of \glspl{nn} is a necessity for various reasons. This paper summarized multiple methods, that aim to visualize these explanations. Since first methods were introduced in \cite{RobnikSikonja.2008} many new approaches have evolved. Some of the most promising methods like \gls{cem} or \todo{was sind denn die interessantesteten sahcxen bei dir?} have found new, but much more interpretable ways of explaining \glspl{nn}. 
\par
Even though there still is no favorable method at this time and for most approaches much more research is necessary, there are more ways than ever to visualize \glspl{nn} in an understandable way.

\section{Future Research}
There have been many new approaches towards Perturbation-Based Methods in the last years. Methods like anchors or \gls{cem} are promising, but need more research in order to refine them and make them applicable to various types of data. Besides, we find that new, reliable and less computational expensive metrics must be a priority for future research.
\todo{ergänzen}