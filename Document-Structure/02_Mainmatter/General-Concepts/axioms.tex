\subsection{Axioms}
Following the terminology of~\cite{Sundararajan.2017}, axioms describe desirable properties which a method should satisfy.

\begin{description}
    \item[\namedlabel{ax:conservativity}{Conservativity}{conservativity}] A heatmapping \(R^{(1)}\) is \textit{conservative} if the sum of assigned relevances in the pixel space corresponds to the total relevance detected by the model.\cite{Montavon.2017} %\todo{this is copied 1:1. bad.}
    Formally, this is expressed by \cref{eq:1}.
    \item[\namedlabel{ax:positivity}{Positivity}{positivity}] A heatmapping \(R^{(1)}\) is \textit{positive} if all values forming the heatmap are greater or equal to zero\cite{Montavon.2017},%\todo{this is copied 1:1. bad.}
    formally
    \[
        \forall i: R_i \geq 0.
    \]
    \item[\namedlabel{ax:consistency}{Consistency}{consistency}] A heatmapping \(R^{(1)}\) is consistent, if it is following both \ref{ax:conservativity} and \ref{ax:positivity}.\cite{Montavon.2017} By this, if there is no detection in the input-sample, the heatmap is also forced to be zero (instead of positive and negative values that cancel each other)\cite{Montavon.2017}.
\end{description}
%
%
%\subsubsection{Continuity}
%\todo{write}
%\subsubsection{Explicitness}
%\citeauthor{AlvarezMelis.2018}
%\todo{write}
%\subsubsection{Stability}
%\citeauthor{AlvarezMelis.2018}
%\todo{write}
%\subsubsection{Diversity}
%\citeauthor{AlvarezMelis.2018}
%\todo{write}
%\subsubsection{Grounding}
%\citeauthor{AlvarezMelis.2018}
%\todo{write}
%\subsubsection{Implementation Invariance}
%\todo{write}
%\subsubsection{Selectivity/Fidelity}
%`Put another way, a method with high fidelity will assign high relevance to %features that, when removed, greatly reduce the DNN’s output confidence in the %class assignment, while assigning low relevance to features that do not greatly %affect the confidence when removed. Fidelity cannot be established %axiomatically, and so must be estimated using implementations of the %explanation method with the DNN model and dataset under investigation.'\cite%{Tomsett.2019} Also look up: \citeauthor{Bach.2015, AlvarezMelis.2018}.