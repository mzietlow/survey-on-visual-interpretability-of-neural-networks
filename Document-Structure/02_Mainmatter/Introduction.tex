\section{Introduction}
More and more \glspl{nn} are used in the day to day life. \glspl{nn} learn something and then replicate it, but how exactly the internals of the network work is not as easy to find out. Finding a way to communicate in an explainable way how \glspl{nn} compute their output has been a problem ever since. 
\par 
Because of that, many different approaches have been made over the last years. Today there are multiple models, which allow a human to understand a \gls{nn}, like heatmaps or diagrams, which show the most important parts of the input.

\subsection{Structure of this paper}
This paper aims to give a brief overview on the current state of research of visualizing a \gls{nn}. 
\par 
Section 1 describes the methodology of the paper, by defining the target audience and finding criteria for inclusion of research. Thereafter definitions for a heatmap, interpretability and explanation are made in section 2. Next in section 3 the Taxonomy for the following chapters is introduced. Then general concepts for understanding the later explained methods are established.
\par
The methods for visualization are categorized in Gradient-Based Methods, Perturbation-Based Methods and Meta-Methods.
Gradient-Based Methods observe the infinitesimal change in model prediction with respect to a root point \(x_0\) of the model \(f\).
Perturbation-Based Methods are based on actually varying the input (by either  perturbation or occlusion) and interpreting the differences in the output.
Meta-Methods use approaches of the other categories and refine them further by combining them oder adding an extra step to the explanation.
\par
Thereafter known issues are addressed.
Finally a conclusion is found and future research is presented.

\subsection{Target audience}
This paper is targeted at an audience that is looking for a theoretical introduction to the field of heatmapping. Knowledge of basic building blocks (such as perceptron, convolutional\=/, recurrent-cells), activation-functions (such as ReLU, tanh), common datasets (such as MNIST, ImageNet), and well known models (such as the VGG-family, AlexNet) is assumed.

\subsection{Criteria for inclusion}
To limit the amount of included research, criteria for inclusion was found. Research which was accepted by a renowned conference or published by a researcher, who is well known for other publications, will be included in this paper. In addition the amount of citations of publications must be significant. Furthermore new research will be mentioned, which was published in the last few months, but not accepted by a conference until now or with just a small number of citations, but shows fundamental new findings.