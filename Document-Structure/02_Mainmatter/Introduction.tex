\section{Introduction}
just to see how long the paper is
\todo{write}

\subsection{Structure of this paper}
This paper aims to give a brief overview on the current state of research of visualizing a neuronal network by heatmapping. Section 1 describes the methodology of the paper, by defining the target audience and finding criteria for inclusion of research.
Thereafter definitions for a Heatmap, Interpretability and Explanation will be made in section 2. Next in section 3 the Taxonomy for the following chapters is introduced. REST FEHLT NOCH


\subsection{Target audience}
This paper is targeted at an audience that is looking for a theoretical introduction to the field of heatmapping. Knowledge of basic building blocks (such as perceptron, convolutional\=/, recurrent-cells), activation-functions (such as ReLU, tanh), common datasets (such as MNIST, ImageNet), and well known models (such as the VGG-family, AlexNet) is assumed.

\subsection{Criteria for inclusion}
To limit the amount of included research, criteria for inclusion was found. Research which was accepted by a renowned conference or published by a researcher, who is well known for other publications, will be included in this paper. In addition the amount of citations of publications must be significant. Furthermore new research will be mentioned, which was published in the last few months, but not accepted by a conference until now or with just a small number of citations, but shows fundamental new findings.